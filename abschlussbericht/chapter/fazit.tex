%!TEX root = ../AutoML-fuer-Segmentierung.tex
\chapter{Gesamtfazit}
\label{ch:conclusion}

%Hier kann beliebig \section{} gesetzt werden


Abschließend können wir nach unserer Arbeit mit den verschiedenen Frameworks sagen, dass sehr große Unterschiede in der Nutzerfreundlichkeit festzustellen sind. Während die Benutzung von NAS-Unet extremst zeitaufwändig und unglaublich schwierig zu erarbeiten war, gestaltete sie sich bei DeepLab deutlich leichter und war bei nnUNet extrem gut machbar. Schade ist dabei, dass uns Auto-DeepLab leider nicht zur Verfügung stand. 
Wir kommen hier zu dem Schluss, dass wir beim nächsten Mal eher auf den Code und die Dokumentation achten und einem Framework wie NAS-Unet nicht so viel Zeit widmen würden. 

Auch im Performancebereich zeigt nnUNet mit Abstand die besten und stabilsten Ergebnisse. Daraus, und aus den sehr guten Erfolgen von nnUnet in den diversen Wettbewerben, folgern wir daher, dass zumindest im Bezug auf medizinische Datensätze, der Ansatz von nnUNet, besonderen Wert auf die Hyperparameter und weniger Wert auf die Architektur zu legen, ein sehr erfolgreicher und weiterzuverfolgender Ansatz ist. Insbesondere im Vergleich zu dem Ansatz von NAS-Unet, der vor Allem Wert auf den Architektursuchraum legt, scheint er eindeutig die bessere Wahl zu sein. 

Abschließend können wir sagen, dass es uns sinnvoll erscheint mit dem Ansatz und den Ideen von nnUNet weiterzuarbeiten und sie eventuell sogar weiter zu entwickeln oder sie auf andere Anwendungsbereiche anzupassen. Letzteres zum Beispiel durch Arbeit an der Kombination der Lossfunktionen. 




% Einträge in die ToDo-Liste setzen (für Änderungen muss 2 mal übersetzt werden, wie bei TOC auch)

