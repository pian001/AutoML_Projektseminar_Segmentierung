%!TEX root = ../AutoML-fuer-Segmentierung.tex
\chapter{Einführung}
\label{ch:intro}

Diese Ausarbeitung ist Gegenstand des Projektseminars ``AutoML``, das im Wintersemester 2020/2021 an der Westfälischen Wilhelms Universität unter der Leitung von Dr. Xiaoyi Jiang und Christof Duhme durchgeführt wurde.\\
In dem Projektseminar haben sich neun Teilnehmer zu drei Dreiergruppen zusammengefunden. Die Themen der drei Gruppen waren: ``AutoML Frameworks``, ``AutoML für Segmentierung`` und ``NAS ohne Training``.\\
Wir haben uns mit Thema zwei ``AutoML für Segmentierung`` beschäftigt und beschreiben im Folgenden unsere Vorgehensweise, Probleme, die wir hatten, Erfolge, die wir erzielen konnten, und gehen zu jedem Framework, das wir genutzt haben, auf die Theorie zu der Architektursuche ein. Insgesamt haben wir uns mit drei Frameworks beschäftigt: NAS-Unet, Auto-DeepLab und nnU-Net. Die Arbeit ist so aufgebaut, dass wir jedem Framework ein Kapitel widmen, in dem wir erst jeweils die Theorie erklären, anschließend von unserer Vorgehensweise mit dem Framework berichten, dann die Ergebnisse einordnen und bewerten und abschließend ein Fazit ziehen. Im Anschluss an diese Einführung werden wir so also in Kapitel 2 über NAS-Unet berichten, Kapitel 3 befasst sich mit Auto-DeepLab, anschließend gehen wir auf das nnU-Net ein, mit welchem wir die mit Abstand besten Erfolge erzielen konnten, wie wir in Kapitel 4 sehen werden, ehe wir in Kapitel 5 ein Gesamtfazit ziehen.\\
Zu jedem Framework haben wir das Ziel verfolgt, die Theorie zu verstehen und das Framework anhand verschiedener Datensätze zu testen und zu bewerten. Die Kapitel spiegelen in ihrer Reihenfolge auch die Reihenfolge wieder, in der wir uns mit den verschiedenen Programmen beschäftigt haben. So haben wir mit NAS-Unet angefangen. Zunächst haben wir versucht, auf unseren eigenen Geräten zu arbeiten. Wir mussten jedoch schnell feststellen, dass NAS-Unet auf Windows nicht unterstützt wird und auf MacOS fehlte uns eine Grafikkarte für die Ausführung. Deswegen haben wir an unseren Windows-Geräten eine Linux Distribution installiert. Doch auch so konnten wir nicht zufriedenstellend arbeiten, da die Architektursuche zu aufwendig ist für unsere Geräte.\\ So haben wir glücklicherweise die Möglichkeit bekommen auf den Universitäts-Servern zu arbeiten. All unser Vorgehen haben wir so auf PALMA II (kurz für ``Paralleles Linux-System für Münsteraner Anwender`` \cite{palmaII}) ausgeführt. PALMA II ist ein \enquote{Computercluster mit über 18.000 Prozessorkernen, deren Zusammenwirken auch die Lösung komplexester Aufgaben aus Wissenschaft und Forschung ermöglicht.} \cite{palmaII}. So hatten wir leistungsstarke Hardware zur Verfügung, auf denen wir die komplexen Berechnungen ausführen konnten.\\
Nachdem die technischen Rahmenbedingungen nun auch erklärt sind, sei noch zu erwähnen, dass dieser Bericht ein (fundiertes) Grundwissen im Bereich Machine Learning voraussetzt.

%Hier kann beliebig \section{} gesetzt werden

% Einträge in die ToDo-Liste setzen (für Änderungen muss 2 mal übersetzt werden, wie bei TOC auch)
\todo{Einführung mit Inhalt füllen}


